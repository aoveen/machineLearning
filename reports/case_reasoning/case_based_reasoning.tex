\documentclass[10pt,a4paper]{article}
\usepackage[margin=1.4cm]{geometry}
\usepackage[latin1]{inputenc}
\usepackage{amsmath}
\usepackage{amsfonts}
\usepackage{amssymb}
\usepackage{multirow}
\usepackage{graphicx}
\usepackage{subcaption}
\usepackage{morefloats}

\setlength{\parskip}{5pt}
\setlength{\parindent}{0pt}

\author{Hector Dearman \and Paul Rowe-White \and Kritaphat Songsriin \and Simon Stuckemann}
\title{Machine Learning CBC: Case-Based Reasoning\\Group 2}
\begin{document}
\maketitle

\section{Case-based Reasoning}
CBR belongs to a specific class of learning algorithm? How are these algorithms called and what are the differences with other learning algorithms, like neural networks and decision trees.

\section{Implementation Details}
Describe implemenation of CBR cases, the retrieve, reuse and retain functions. Flowchart of how the code works.

Describe how you initialise your CBR system.

\subsection{Dealing with Existing Cases}
Discuss what happens if you try to add a case to the CBR system that is already there (either in the initialisation phase or in the retain function) How did you deal with this issue?

\subsection{Similarity Measures}
Compare the different similarity measures you have used (at least three). What are the advantages/ disadvantages of each measure?

\subsection{Dealing with Multiple Best Matches}
How did you solve the problem of finding two or more best matches with different labels in function RETRIEVE


\section{Performance Results}

Is there any difference in the CBR performance on noisy and clean data. If yes, why?

Commented results of the average confusion matrix, average classification rate and recall, precision and F1 measures per class.
\end{document}
