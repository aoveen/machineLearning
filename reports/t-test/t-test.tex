\documentclass[10pt,a4paper]{article}
\usepackage[margin=1in]{geometry}
\usepackage[latin1]{inputenc}
\usepackage{amsmath}
\usepackage{amsfonts}
\usepackage{amssymb}
\usepackage{multirow}
\usepackage{graphicx}
\usepackage{subcaption}
\usepackage{morefloats}

\setlength{\parskip}{5pt}
%\setlength{\parindent}{0pt}

\author{Hector Dearman \and Paul Rowe-White \and Kritaphat Songsriin \and Simon Stuckemann}
\title{Machine Learning CBC: Comparison of Machine Learning Algorithms\\Group 2}
\begin{document}
\maketitle

\section{Introduction}
Even though this report does not contain any t-test results, we would use the paired t-test in order to compare the different classifiers. This is because to evaluate the classifiers' performance, we used K-fold cross validation (in particular, we used $K=10$) with the same folds for all classifiers. Therefore, over several folds, we evaluate the performance of our classifiers on the exact same data. This is important since if the classifiers were trained on separate data samples the difference in error could be partially attributable to differences in the makeup of the two samples since the different samples are not independent \cite{Mitchell:1997:ML:541177}. In contrast, using the same folds to measure the performance for all classifiers, we are able to use the \emph{paired t-test} which typically produces tighter confidence intervals because all differences in observed errors in a paired test are due to differences between the hypotheses \cite{Mitchell:1997:ML:541177}.

Note that in this exercise the t-test is performed on the classification rates. However, other measures to evaluate the performance of a classifier exist. In fact, we are calculating the $F_1$ measure per class in every fold. However, note that one of the t-test's requirements is that each of the populations being compared follow the normal distribution. It turns out that the error rate/classification rate follow are indeed normally distributed, whereas the $F_1$ measures do not (given by the \emph{Shapiro-Wilk} test).

\section{Test Results}
Table \ref{tab:classificationRates} lists the classification rates for the deduction of 6 basic emotions using three different machine learning algorithms. Note that we implemented two different different classifiers using artificial neural networks and we were unable to make a final conclusion as to which classifier performed better. Both versions are therefore included here.
\begin{table}[!ht]
\centering
\begin{tabular}{|c|c|c|c|c|}
\hline 
 & \textbf{Decision Trees} & \textbf{six-output NN} & \textbf{six single-output NN} & \textbf{CBR} \\ 
\hline 
\textbf{Clean Data} & $0.7280$ & $0.8180$ & $0.8330$ &  $0.8700$\\ 
\hline 
\textbf{Noisy Data} & $0.5930$ & $0.7350$ & $0.7090$ &  $0.7550$\\ 
\hline 
\end{tabular}
\caption{Classification Rates of Performance of Learning Algorithms for Deducing Emotions from Facial Expressions}
\label{tab:classificationRates}
\end{table}

Assuming the differences in classification rate laid out in Table \ref{tab:classificationRates} are statistically significant, the data strongly suggest that the \emph{decision tree} learning algorithm performs by far the worst when classifying emotions from facial expressions. We can see that the classification rate for clean data ($0.7280$) is actually similar to the classification rates of the performance of both neural networks algorithms on the noisy data. The performance of the \emph{decision tree} learning algorithm with $0.5930$ is even worse.

As discussed in the previous reports there is no significant difference between the classification rates of the two versions of the \emph{neural network} learning algorithm. However, the data here suggests that the both \emph{neural network} algorithms perform better than the decision tree algorithm. However, compared to the high performance of the \emph{CBR} approach a performance of approx $80\%$ and $70\%$ for clean and noisy data respectively. Interestingly, this data suggests that neural networks are less susceptible to changes in the data as the performance drop on noisy data is not as significant as that for decision trees. The \emph{case-based reasoning} learning algorithm appears to perform best for learning emotions from facial expression data. The classification rate for the clean dataset is $87\%$ which is relatively good compared to the other algorithms. Also, note that the performance on the noisy data drops similarly to the performance drop in the \emph{artificial neural network} algorithms. However, the performance on noisy data is still higher and, interestingly, also higher than the performance of the \emph{decision tree} learning algorithm on the clean data.

Note that the above results only apply in this particular example. We therefore cannot claim that some algorithms are better learning algorithms in general since the performance of a machine learning algorithm very much depends on its application domain. Different algorithms have different applications and are favourable in different situations. In other words, in order to achieve a high accuracy with a machine learning algorithm, this algorithm has to be able to fit the data well. Since no data is the same there is not a single algorithm that always performs better than others. For example, it turns out that the poorly-performing \emph{decision tree} learning algorithm in our case is actually used in practice and performs very well in \emph{Microsoft}'s \emph{Kinect} platform, to recognising body parts \cite{Shotton2011}.

\section{Potential Changes to the Learning Algorithms}
\subsection{Varying the Number of Folds}
If the value of K is increased, the number of examples per fold decreases. This means that the size of the training set is closer to the size of the entire data set which in turn means that there is less (pessimistic) bias towards overestimating the true expected error. At the same time though, more examples in the training set mean that there are less examples in the test set. A small number of examples in the test set means that the test set does not represent the real distribution of the data as accurately as before, and thus results in higher variance of the error.

Therefore, there are two different sources for error: variance of the error and also bias of the error. In order to reduce both the bias and the variance of the error the number of examples has to be increase, since the two measures are anti-proportional to each other. In practice, this is often not possible which is why when performing K-fold cross validation one has to make a trade-off between the two sources of error. 

It is easy to recognise that this behaviour of the two sources of error is closely related to the problem of overfitting. As the error due to bias decreases, the error due to variance increases. The point at which the classifier is neither over- nor underfitting is where the increase in variance is equivalent to the reduction in bias.

This means that in order to choose K in K-fold cross validation one has to attempt to find this exact spot.

\subsection{Adding Emotions}
As discussed in the \emph{Case-based reasoning} report, there are two types of machine learning algorithms: \emph{eager learning algorithms} and \emph{lazy learning algorithms}. 

Eager learning algorithms attempt to find a general approximation of a target function from a set of examples that are given to the learning algorithm as its initial input. \emph{Decision Trees} and \emph{Artificial Neural Networks} are examples of the \emph{eager learning method} as both algorithms find an approximation of the general target function from the example data and are only then ready to classify new instances. This approach means that in order to add additional emotions to the set the learning algorithm would have to find a new approximation of the general target function. From an engineering perspective this means that decision trees would again have to be built using the dataset with a new set of emotions. Note though, that this is a relatively fast process which means that even though the classifier has to be built again, this is possible in a relatively short amount of time.

In contrast, artificial neural networks are built using several parameters. These parameters are optimised using the given data and the actual classifications of the examples that are fed into the optimisation process. Since this is essentially a very large search problem this process can take a long time to complete which means that \emph{artificial neural networks} are by far be the least suitable method for incorporating the new classes.

On the other hand, \emph{lazy learning algorithms} do not attempt to find an approximation of a general target function up front. Instead they simply store past examples and assign an approximate target function to a new instance on the fly based on older examples. This means that \emph{lazy learning methods} adapt very well to change which means that adding new emotions to the classifier is as simple as feeding the algorithm examples of those new emotions.

\bibliography{t-test}
\bibliographystyle{plain}

\end{document}
